\section{Ondorioak}
\subsection{Informazio teknologiak}
Informazio teknologiei dagokionez, zailtasun handiena ITILeko praktika onak gure proiektu honetara era zehatz batean inplementatzea izan da, baina behin aplikaturik dagoela, konturatu gara asko errazten duela IT teknologien eraginkortasuna enpresan. Gure kasuan, inzidentzia guztiak leku zentralizatu batean jarriaz, IT departamentuari lana asko erraztuz eta erabiltzaileei begira seriotasun irudia emanez. Horrelako webgune bat izateak, telefonoak eta antzeko baliabideek ez bezala, asko handitzen du eskalagarritasuna eta hainbat eta hainbat erabiltzaile egon daitezke inzidentziak sartzen webguneak. Gero kontuan izan behar da, noski, gero IT departamentuak ere izan behar dituela baliabideak inzidentziei aurre egiteko, notifikazio zerbitzua, aurpegia besterik ez baita.


\subsection{Metodologia}
Hainbat dokumentu sortu dira, garapenari laguntzeko. Horietako bat izan da kodifikazio manuala, nola kodifikatu behar den adieraziz. Erabilgarria izan arren, badaude hobetu ahalko litzatekeen gauza nagusi bat, esplizitua izatea. Bertan datozen gauza asko, lengoai askotan erabili daitezkeenak dira, baina beharrezkoa da Javan eta bereziki JSPko gauza propioetan direnak (beste lengoaiek ez dituzten propietate edo sintaxia) gehiago sakontzea, gure produktua hobetzeko.

Bertsio kontrolari dagokionez, asko erraztu du hainbat pertsonen artean aplikazioa garatzea, eta autokritikarako ere tresna egokia izan da (noiz egin diren aurrerapen handienak, noiz gutxien). Oro har, esan daiteke behar-beharrezko tresna bihurtu dela taldean lan egiteko.

Berrikuspen eta frogekin, esan behar da berrikuspenak era informalean egin direla baina ez dela egon aparteko txostenik, egin den gauza bakarra izan da, hurbiltasuna profitatuz, garatzaileari galdetu ea zertarako zen gauza bat edo bestea, kasu batzuetan zuzenketak eginez. Frogak aplikatu egin dira eta ez da egon inolako arazorik, behintzat txosten hau bukatu den egunera arte.

\subsection{Segurtasuna}
Firewall eta VPNen erabilgarritasuna konprobatu da, erregela desberdinek trafikoari nola eragiten dion ikusita. VPNaren kasuan, zifraketaren balioa ere kontuan hartu da, birtualki, oso urrun egon daitekeen pertsona bati gure sarean bertan egotearen abantaila guztiak emateak dituenak, hain zuzen, zifraturiko konexioak hori ahalbidetzen duelako. VPNak eskaintzen duen alternatiba merkea ere baloratu da, puntutik punturako konexioekin alderatuz gero.
