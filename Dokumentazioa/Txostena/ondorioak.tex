\section{Conclusions and future improvements}
%optimizazioa hari globalak seederrak txekatzeko, deskargak bertan registratu eta guztiei eramateko erantzunak. eta gordetzailearekin berdin
%Deskargak gelditzea

The main conclusion draw is that modelling the IDL file correctly at the beginning, analysing the functionalities and structure of the program beforehand, is crucial to have good rhythm of work. Changing the IDL and thus, the main structure of the program after starting the development, has very negative consequences in the course of development. We started later developing, because of multiple changes in the IDL and design of the program. But, in the long run, we gained many hours.

As far as future improvements are concerned, we have outlined two: there is no functionality of pausing or removing current downloads and it would be very interesting to have them, as it is one of the functionalities most Peer to Peer programs have. It would also improve user experience, because once a download starts, the user has no power over it until it is finished.

The other main improvement has little to do with user experience. It is closely related to performance. Our program uses multiple threads that eventually can lead to heavy CPU usage. Reducing CPU use is a priority if we want to be possible to use the program along with others comfortably. One partial simple solution we propose is to use a single \texttt{SeedChecker} to all of the downloads, where \texttt{Deskargak} would register and the checker would go individually over them. The same could be done with \texttt{Gordetzaileak}. That way, ther would be no extra threads per download and we think performance should improve, at least to a noticeable level, once the download number grows.