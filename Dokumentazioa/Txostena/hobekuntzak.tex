\section{Hobekuntzak}
Hobekuntza posibleen artean nabarmendu daitezke eginiko aplikazioaren alorrean, CMDB-aren diseinu hobea (horrenbeste eremu nulo izango ez lituzkeena) edota inzidentzietaz harago doan kudeaketa eredu osoaren inplementazioa, ITILek eskaintzen duen bibliografia eta lan-tresna mardulak profitatuz.

Era berean, segurtasunari dagokionez, Firewall-aren diseinu errealistagoa egin daiteke, DMZtan beste zerbitzari batzuk jarriaz eta bertara kanpoko sarbidea baimenduz (web eta posta zerbitzariak, akaso). Era berean, barne saretik kanporako trafiko osoa baimenduta dagoen aldetik (web filtroekin, egia da), agian hori findu beharko litzateke eta konexio mota batzuk ez baimendu, segurtasun neurriak zorrozteko. Datu-basean, \textit{hash} algoritmo indartsuago bat erabiltzea ere ondo legoke, pasahitzen krakeoa zailtzeko.

Azkenik, erabilitako metodologiarekin, alderik nabarmenena, Github erabiltzerakoan erabiltzaile bakarra ez erabiltzea da, baizik eta garatzaile bakoitzak bere izena edo izenordaina erabiltzea, jakiteko zehatz-mehatz nork egin duen zer. Oraingo metodoarekin, talde guztiak dauka erantzukizuna, eta ez da maila pertsonalera heltzen.
