\section*{Appendices}
\addcontentsline{toc}{section}{Appendices}

\section{Design patterns}
In software engineering, a design pattern is a general, repeatable and language-independent solution to a commonly occurring problem in software design. They allow people to communicate more easily, because programmers can use well known names and software interactions, and to understand exchanged code better. They also speed up the development process by providing tested efficient solutions to common programming problems.

Design patterns are divided into different groups depending on their purpose, creational, structural or behavioral are only some of them.

\subsection*{Creational design patterns}

The design patterns on this category concentrate on the creation and initialization of different objects.
One of the most used design patterns falls into this category, the singleton. This pattern is used when an instance of an object has to be used through all the program. We use this pattern when interacting with the different elements of the ORB and the server via the Globalak class.
\begin{lstlisting}[language=java]
public class A {
    private static A a;
   
    private A(){
        init();
    }

    public static A getA(){
        if(a == null) A = new A();
        return a;
    }
}
\end{lstlisting}
In this pattern the only way to use the \lstinline{a} field is calling the  \lstinline{getA()} function, and the constructor and the field being \lstinline{private} ensures that there is only one instance of \lstinline{A}.

Different factories are also in this category.


\subsection*{Structural design patterns}

These patterns are about class and object composition and how they interact between them using interfaces to obtain new functionalities.

The java graphical user interface framework, Swing, uses these patterns to create and manage different elements.


\subsection*{Behavioral design patterns}

This category concentrates on the communication between different objects.

In this category is the iterator that lets us iterate on a list.


\subsection*{Concurrency design patterns}

The patterns on this category deal with multi-threaded programming problems.

The thread pool or the different kinds of using shared resources fall into this category

\section{Bibliography}

\noindent THE JACORB TEAM. \textit{JacORB 2.1 Programming Guide} [online]. The JacORB Team, 2004. \newline \htmladdnormallink{http://jmvanel.free.fr/corba/doc/jacorb-ProgrammingGuide.pdf}{http://jmvanel.free.fr/corba/doc/jacorb-ProgrammingGuide.pdf} [Last consultation: 6-13-2011]
\bigskip

\noindent GITHUB. \textit{GitHub:Help} [online]. GitHub, 2011. \newline \htmladdnormallink{http://help.github.com/}{http://help.github.com/} [Last consultation: 6-13-2011]
\bigskip

\noindent GAMMA, Erich; HELM, Richard; JOHNSON, Ralf; VLISSIDES, John M. \textit{Design Patterns: Elements of Reusable Object-Oriented Software}. Boston: Addison-Wesley Professional, 1994. \newline 
\bigskip

