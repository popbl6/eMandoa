\section*{Sarrera}
\addcontentsline{toc}{section}{Sarrera}

Lan honetan, enpresa fiktizio batean zerbitzu seguru bat jartzen da, inzidentzia desberdinak era eraginkor eta seguru batean jasotzeko.

Enpresa fiktizio hori, hezkuntzarekin loturikoa izango da, proiektua hasi baino lehen eginiko lanetan egindako ikerketak berrerabili ahal izateko eta gainera, ez duelako horrenbesteko eraginik proiektuaren funtsan.

Eraginkortasuna aipatzen da, industrian hainbatetan erreferente moduan ikusi den ITIL prozesuei jarraiki egin baita, enpresa-arkitekturarekin lerratuz IT teknologiak, nahiz eta inzidentzien bakarrik loturiko lagin txikia izanik. 

Segurtasuna, aldiz, aplikazioaren eta bere datuen konfidentzialtasuna, eskuragarritasuna eta integritatea bermatzeko erabiliko da, sarea firewall baten bitartez babestuz eta kanpoko konexioen konfidentzialtasuna bermatuz, kontrako oso faktore gutxirekin, sare pribatu birtualen bitartez.

Azkenik, esparru horietan lan egiteko metodologiak ere garrantzi handia izango du, garaturiko produktuaren kalitate teknikoa (ez funtzionalitate aldetik) baino gehiago. Beraz, lan egiteko erak izango du protagonismo handiena, softwareari dagokionez.

Hurrengo orrietan, proiektu eta produktuaren deskribapen osoa egingo da, bai garaturiko proiektuarena, baita garapen horren prozesuan emandako pausu guztiak ere. 

\begin{flushright}
Egileek, Arrasaten, \date{\today}an.
\end{flushright}
