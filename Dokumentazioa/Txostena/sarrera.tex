\section*{Introduction}
\addcontentsline{toc}{section}{Sarrera}

A P2P program, essentially, is an application that is used to share content through the network. Its most important difference from other downloading methods (FTP servers etc.) is that there is no central node in which content is downloaded from and uploaded to. The content flows through a decentralized network of nodes, clients, who simultaneously work as servers (or seeders, because they seed content) and downloaders (leechers). The most famous of them is eMule or aMule (depending on the Operating system used).

To illustrate different steps in the development phase, different kinds of diagrams have been used, starting from the user case diagrams in the very beginning until the fully-detailed class-diagram, all of them constructed using UML (the Unified Modelling Language). To make possible having different ends working at the same time, the resulting application is a distributed one, built using a Corba (Common Object Request Broker Architecture) implementation in the Java language, JacORB.


\begin{flushright}
The authors, Arrasaten, in Arrasate the \date{\today}.
\end{flushright}
